% SPDX-FileCopyrightText: Copyright (c) 2024-2025 Yegor Bugayenko
% SPDX-License-Identifier: MIT

\documentclass{main}
\pagecolor{white}
\usepackage[american]{babel}
\title{\includegraphics[scale=0.5]{../images/zerocracy-logo.pdf}\\
  Zerocracy: A Project Manager That Never Sleeps\\
  {\small\colorbox{blue!20!black}{\color{white}{Executive Summary}}}}
\author{Yegor Bugayenko\\
  \texttt{yegor256@gmail.com}\\
  \href{https://www.zerocracy.com}{\texttt{www.zerocracy.com}}\\[1em]
  \href{https://github.com/zold-io/papers/releases/tag/\zoldversion}{\texttt{\zoldversion}}}
\begin{document}
\date{July 6, 2019}
\maketitle

\begin{abstract}
Lack of systematic, accurate and precise project management based on data and
analysis is the main failure factor in software development projects. Project
managers can't do this job right, because it is routine, boring and requires a
lot of time. Thanks to the growing popularity of virtual collaboration tools and
emerging power of AI there is a great opportunity for Zerocracy to improve the
human workforce with computers. The target market is very similar to one
acquired by Atlassian over the last 15 years: it includes 40,000 companies and 5
million software developers. Using our strengths and a unique market opportunity
we plan to create a new segment of AI project managers in the RPA market, engage
20,000 customers, and generate revenue of \$400M.
\end{abstract}

\section{Market Analysis}

According to
\href{https://www.infoq.com/articles/software-failure-reasons}{The Most Common Reasons Why Software Projects Fail}
article in InfoQ
(July 2015), most of the problems in software projects come from: 1) Lack of
substantial data and analysis; 2) Excessive personnel, added because of
unrealistic predictions; 3) Inability to adjust budget and time forecasts to
changing requirements; and 4) Ignoring of facts and statistics.

It is obvious that the first problem (lack of data and analysis) is the root
cause of all others. The management doesn't have enough information when needed:
that's why it can't make appropriate decisions and projects experience problems.

\href{https://www.projectsmart.co.uk/white-papers/chaos-report.pdf}{Chaos Report} (2015)
by Standish Group says that ``Software development projects
are in chaos, and we can no longer imitate the three monkeys---hear no failures,
see no failures, speak no failures.'' The report also demonstrates that as a
result of this chaos we have restarts (94\% projects!), cost overruns, and time
overruns. Technology incompetence is the root cause of project failures only in
7\% cases. In almost all other cases the management is the source of trouble.

Thus, key pain point of software projects all over the world is lack of proper
management, which should be based on systematic, accurate, and precise data
collection and analysis.

Our customers are companies that run software projects.

Our market consists of the labor, tools, and services that help companies run
their projects.

\subsection{Competition}

``Labor'' is the biggest category of competition: our customers hire people  in
order to manage their software projects, including Scrum masters, technical
leads, project managers, program managers and everybody else who manage
programmers. This \href{https://www.binfire.com/blog/2016/06/how-many-project-managers-in-the-world/}{simple analysis}
of LinkedIn data demonstrates that there are
over two million managers in the world (although, not all of them manage
software projects). We can assume that our customers spend about \$100B on
salaries of project managers (\$50K salary for an average manager).

``Tools'' is the second category of competition, it includes task trackers,
pla\-n\-ners, and project management software. This
\href{http://www.marketsandmarkets.com/Market-Reports/project-portfolio-management-software-market-225932595.html}{report}
shows that the size of
project portfolio management market is \$2.5B (although, it includes services
too). This \href{http://www.idc.com/getdoc.jsp?containerId=256779}{report}
by IDC says that it is \$3.7B. This
\href{https://www.capterra.com/project-management-software/user-research/}{report} says that 7\% of this
market belongs to software development projects. The most popular project
management software systems are
\href{https://products.office.com/en-us/project/project-and-portfolio-management-software}{MS Project},
\href{https://www.atlassian.com/software}{Atlassian product family},
\href{https://podio.com/site/en}{Podio},
\href{https://www.wrike.com/}{Wrike},
\href{https://basecamp.com/}{Basecamp}, and
\href{https://www.capterra.com/project-management-software/#infographic}{others}.

``Services'' is the third competitive category that includes management
consulting, coaching, training, Agile certifications, and many other forms of
indirect management of programmers. Bloomberg
\href{http://www.bloomberg.com/news/articles/2013-06-13/where-the-growth-is-in-management-consulting}{said}
in 2013 that the size of this
market is \$39B.

There is no such thing on the market as robotic project managers at the moment.

\subsection{Market Size}

This \href{www.capterra.com/project-management-software/#infographic}{report}
says that \href{https://www.atlassian.com/}{Atlassian}
(NASDAQ:TEAM) has 35,000 paying customers. Their
own \href{https://blogs.atlassian.com/2014/11/atlassian-customers-thank-you/}{blog post}
confirms that they have over 40,000 customers, acquired in less
than 14 years.

ICT \href{https://www.infoq.com/news/2014/01/IDC-software-developers}{says}
that there are 11 million professional software developers in the
world. At the same time, there are \href{https://sostats.github.io/}{about five million accounts} registered at
StackExchange platform, which is the dominating web resource for programmers,
where they ask and answer technical questions. We assume that 5 million is the
amount of programmers in the world who make a living by regularly writing code.

\subsection{Segmentation}

While most software companies actively exploit the concept of outsourcing and
offshoring in order to optimize development costs, according to Yegor's blog
traffic \href{https://twitter.com/yegor256/status/770780702711226369}{statistics}
programmers are geographically located in the United States
(24\%), India (7\%), Germany (6\%), UK (6\%), Russia (5\%), and others.

StackOverflow \href{https://stackoverflow.com/research/developer-survey-2016}{annual survey}
demonstrates that about 12\% of developers  work
remote full-time. 30\% work remote part-time or full-time; developers with 11+
years experience are nearly twice as likely to work remote as developers with
less than 5 years experience.

LifeHacker \href{https://lifehacker.com/the-best-alternatives-to-google-code-for-your-programmi-1691688947}{claims}
that ``GitHub is the juggernaut in this arena, obviously, and
the web's most popular code repository,'' while others are
\href{https://www.codeplex.com/}{CodePlex},
\href{https://bitbucket.org/}{BitBucket},
LaunchPad, and
\href{https://sourceforge.net/}{SourceForge}.
GitHub \href{https://github.com/blog/1724-10-million-repositories}{blog} announced 10 million repositories
milestone in 2013.

\subsection{Market Trends}

Atlassian \href{https://blogs.atlassian.com/2016/03/software-development-trends-2016/}{blog post}
notes that going remote with smaller teams (10 developers or
even less) is the trend, among a few others. Gallup confirms, according to this
\href{https://www.forbes.com/sites/davidsturt/2014/05/14/working-remotely-does-the-research-prove-it-wont-work-for-you/#9a6518966ac4}{Forbes article},
that ``nearly 4 out of 10 companies currently allow some
employees to work remotely''.

The software development industry is growing every year: by 4.5\% every year
according to CompTIA, by 6\% according to \href{http://www.computerworld.com/article/2502348/it-management/it-jobs-will-grow-22--through-2020--says-u-s-.html}{ComputerWorld},
and even faster
according to \href{http://venturebeat.com/2013/10/17/listen-up-investors-the-software-industry-is-growing-quickly-and-in-unexpected-places/}{VentureBeat}.

Transparency Market Research in their \href{http://www.transparencymarketresearch.com/it-robotic-automation-market.html}{recent report} about RPA market trends
notes that ``automation is soon expected to become a game changing technology in
the transformation of IT industry; the notion that robotic software can
eliminate the need to offshore and at the same time lead to highly automated
efficiency has captured the attention of a large number of IT players, globally;
business process outsourcing (BPO) is one of the key segments in the IT industry
where high adoption of robotic process automation is anticipated in the coming
two to three years.''

\section{SWOT}

There are a number of strengths and weaknesses in our product. Also, there are
some opportunities and threats on the market.

\subsection{Strengths}

The idea of a robotic project manager is \emph{unique} and hasn't been implemented yet
by anyone on the \href{https://en.wikipedia.org/wiki/Robotic_Process_Automation}{Robotic Process Automation} (RPA) market.

It's a fixed-cost business since the development expenses don't change
with the growth of the user base.

The concept was tested with 40+ projects and 350+ programmers.

We have a patent application for it:
\href{https://patents.google.com/patent/US20110196798}{US 12/703,202}.

\subsection{Weaknesses}

The uniqueness of Zerocracy management model and it's strict focus on quality
and results makes it's difficult to engage a large amount of programmers. Thanks
to the large market of traditional working models (pay by time), very few
programmers are interested in working with Zerocracy if their rates are
similar to the ones they earn in their full-time jobs. However, it's impossible
to pay them 3+ times more, because there is no strong customer base as of yet.

\subsection{Opportunities}

We may acquire the entire emerging market, since we're the first player.

Since the lack of proper project management is a constantly growing concern in
software projects, addressing it now with AI may put us in a driver's seat for
the entire software industry.

\subsection{Threats}

Competitors may quickly catch up, since the concept is very visible and
transparent since its market launch.

\section{Team}

Yegor Bugayenko (CEO) \href{http://www.yegor256.com}{blog} has over 50,000/mo unique visitors, most of them are from
software industry: programmers, managers, founders. Besides that, Yegor
Bugayenko is an author of \href{https://amzn.to/2bYXQy7}{Elegant Objects} books about object-oriented
programming, a \href{https://lanyrd.com/profile/yegor256/}{regular speaker} at software conferences, and an active social
networker (12K+ Twitter followers, was recently mentioned by \href{https://techbeacon.com/java-leaders-you-should-follow-twitter}{TechBeacon} as one
of 39 Java leaders and experts to follow on Twitter). The
early adoption of Zerocracy comes from this audience.

\href{https://www.linkedin.com/in/erik-larson-b287ba9}{Erik J. Larson} (scientific adviser) has over a decade of experience in
scientific research on AI, with a focus on dialogue systems and natural language
processing, central to Zerocracy core technology. Larson is also a
\href{https://www.theatlantic.com/technology/archive/2015/05/the-humanists-paradox/391622/}{writer} and
speaker on issues in AI, and will be active in evangelizing Zerocracy
technology.

\href{http://www.yegor256.com/about-me.html}{Yegor Bugayenko} possesses project management certifications, including PMP,
PRINCE2, MSF, RUP, and COSMIC. This means that he has a strong background in
project management and understands its problems, risks and opportunities.

Yegor has some experience in making products for software industry. For example,
there are few of his recent ``pet'' projects:
\href{http://www.rultor.com}{rultor.com} (300+ customers),
\href{http://www.s3auth.com}{s3auth.com} (1200+ customers),
\href{http://www.jare.io}{jare.io} (100+ customers).


\section{Objectives}

Our objective is to build a market of robotic project management and acquire its
majority. Our long-term goal estimate is 20,000 customers in eight years.

Financial objective is \$400M revenue with profit margin over 80\%. Such a high
margin can be achieved due to our AI-as-a-service business model, where growth
of the customer base has almost no impact on the structure and volume of our
fixed costs.

Atlassian is a good case study for us. Their revenue was \$457M in the last 12
months. They are a public company (on NASDAQ since 2015). Their market cap is
\$6.6B. They are our indirect competitors, which means that their clients can
become our customers without leaving Jira.

Short term technical objectives include the following features:

\begin{itemize}\itemsep0em
  \item More sophisticated natural language processing (NLP) in the chat bot;
  \item Proper risk analysis;
  \item Integration with other task trackers, e.g. Jira, and Trello;
  \item More prediction algorithms with more metrics;
  \item Direct access to the pool of freelancers (recruit-as-you-go);
  \item Change request instant estimates;
  \item Benchmarking;
  \item ISO-9001, ISO-27000, and CMMI certifications.
\end{itemize}

Long term goal is to expand our solution to other verticals, including
construction, healthcare, education, etc. Boldly, we estimate the entire
potential of the market at \$100B, when all routine project management roles will
be filled by AI.

\section{Marketing Strategy}

Our USP is: ``a project manager that never sleeps.''

Our market positioning will have four anchoring points:

\begin{itemize}\itemsep0em
  \item ``Friendly'': it is a chat bot that helps us coordinate ourselves;
  \item ``Adaptive'': it immediately re-builds plans when requirements change;
  \item ``Smart'': it predicts problems even before we can think about them;
  \item ``Objective'': its decisions are based on data, not emotions.
\end{itemize}

The list of benefits a customer gets from our solution:

\begin{itemize}\itemsep0em
  \item Cost and schedule overruns are eliminated;
  \item Project schedule and budget are visible and updated instantly;
  \item Developers have clear plans and instructions;
  \item Management always have enough information to make decisions;
  \item Productivity of programmers increases;
  \item Staff turnover decreases;
  \item Cost of management is minimized.
\end{itemize}

The cost of hiring a Zerocracy is proportional to the number of tasks it is
managing in a particular project: ``pay-per-task'' (PPT) cost model with
a fixed price of \$4.00/task. According to our experience, in an average project an average
programmer completes 50 tasks per month, provided an average task takes
approximately 2 hours of work. Thus, a project of ten people, at its peak
performance, completes 500 tasks per month. The cost of Zerocracy in this
scenario is thus \$2,000.

This is what an average software team of 10 people (USA, Western Europe, distributed) spends annually:

\begin{tabular}{lr}
\hline
Expenses & Annually \\
\hline
Payroll (\$60K per person) & \$600,000 \\
Project manager & \$60,000 \\
W2 taxes, fringe benefits (20\% of payroll) & \$132,000 \\
Office, computers, office expenses & \$60,000 \\
Recruiting & \$120,000 \\
Management consulting, trainings, coaching & \$15,000 \\
Servers, cloud, etc. & \$12,000 \\
GitHub private repositories & \$1,200 \\
MS Project license (2 years term) & \$550 \\
JIRA license & \$1,000 \\
\hline
\end{tabular}
\vspace{1em}

An immediate financial effect for our customers would be minimization of labor
expenses (\$60K), entire removal of management consulting (\$15K) and project
management tools (\$550). Thus, Zerocracy will cost them \$24K, while their
savings will be over \$75K (even though this is not our primary selling point).

We don't expect our customers to eliminate their project managers. Instead, we
expect them to transfer personnel to more creative roles, like requirements
analysis, product validation, strategic planning, etc.

Our promotion strategy includes:

\begin{itemize}\itemsep0em
  \item Consumer training programs;
  \item StackOverflow banner ads;
  \item Twitter promoted ads;
  \item Reddit campaigns;
  \item Direct door-to-door sales (enterprise customers);
  \item User conferences;
  \item Free on-site and online consultation sessions;
  \item Free webinars;
  \item Sponsorship of software and management events;
  \item Partnerships with GitHub, Bitbucket and similar platforms;
  \item Benchmarking competitions among champion customers;
  \item Grants and donations to young programmers and their projects;
  \item Blogs with analysis and statistics of our AI software.
\end{itemize}

\section{Financials}

Before a programmer is ready to join a real project, he or she has to be
trained in a ``sandbox'' project. According to our current \href{http://www.zerocracy.com/policy.html}{Policy},
the programmer has to earn 1024 reputation points in order to ``graduate.''
One reputation point is given for each minute earned. With the attractive
hourly rate of \$50, and an average rejection rate of 70\% (this is how many
people quit the platform because they can't put up with our strict requirements),
the ``acquisition cost'' of one developer is close to \$3,000.

A motivated and graduated programmer at an average pace can close 8 tickets per day,
working 20 days a months. This means \$4,000 monthly income for a programmer
and \$1,280 revenue for Zerocracy (we will charge \$8 per ticket).

Thus, every \$100K invested into sandbox projects produce 30 graduated programmers,
ready to work in real projects and generate \$40K per month in revenue.

\end{document}
