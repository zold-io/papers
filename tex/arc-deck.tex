% SPDX-FileCopyrightText: Copyright (c) 2024-2025 Yegor Bugayenko
% SPDX-License-Identifier: MIT

\documentclass{deck}
\usepackage[american]{babel}
\usepackage{merriweather}
\begin{document}
\date{July 6, 2019}

\headslide{Zerocracy Architecture}

\slide{%
  Key Components
  \vspace{1em}

  \begin{tikzpicture}
    \component{claims}{8,8}{Claims}
    \component{radar}{8,12}{Radar}
    \draw [ln] (radar) -- (claims);
    \component[fill=white]{tts}{2,11}{TTS}
    \draw [ln] (tts) -- (radar);
    \actor{user}{-2,8}{User}
    \draw [ln] (user) -- (tts);
    \component{farm}{8,4}{Farm}
    \draw [ln] (claims) -- (farm);
    \draw [ln] (farm) -- (tts);
    \component{xocuments}{14,4}{Xocuments}
    \draw [ln] (farm) -- (xocuments);
    \component{dash}{14,8}{Dashboard}
    \draw [ln] (dash) -- (xocuments);
    \draw [ln] (dash) -- (claims);
    \component{policy}{3,5}{Policy}
    \draw [ln] (farm) -- (policy);
  \end{tikzpicture}
}

\slide{%
  Claims

  \begin{multicols}{2}
  \begin{tikzpicture}[scale=0.75, every node/.style={scale=0.75}]
    \component{queue}{8,8}{SQS Queue}
    \component{sanitizer}{8,12}{Sanitizer}
    \draw [ln] (sanitizer) -- (queue);
    \component{dispatcher}{8,4}{Dispatcher}
    \draw [ln] (queue) -- (dispatcher);
    \component{footprint}{2,6}{Footprint}
    \component{logs}{3,2}{Logs}
    \component{farm}{8,0}{Farm}
    \draw [ln] (dispatcher) -- (farm);
    \draw [ln] (dispatcher) -- (logs);
    \draw [ln] (dispatcher) -- (footprint);
  \end{tikzpicture}
  \vfill
  \columnbreak
  \begin{multicols}{2}
  \topic{zblue}{Java}{%
    The main programming language of the system is Java 8. The build
    system is Maven 3. The deployment is automated by \href{http://www.rultor.com}{Rultor}.
    Continuous integration is done by \href{https://travis-ci.org}{Travis}.
    Static analysis is performed by \href{http://www.qulice.com}{Qulice},
    an aggregator of Checkstyle and PMD.}
  \topic{zblue}{SQS Queue}{%
    It is a large global cross-projects queue of claims, maintained
    in the AWS SQS. Anyone can drop
    a claim into the queue, it is a very fast operation. The processing
    of the claims happen later, when the dispatcher is up to it.}
  \topic{zblue}{XML}{%
    Each claim is a short XML document with a pre-defined
    structure, validated against a XSD Schema. It includes information
    about the source of the event, the time and date, the type of
    request, and the list of supplementary arguments.}
  \topic{zblue}{Footprint}{%
    There is a MongoDB database with a single table that records every
    claim that has been seen in the queue. Later, the footprint is
    accessible via the Dashboard, for monitoring and tracking purposes.
    The database is backed up every hour/day/week via \href{https://www.threecopies.com}{ThreeCopies}.}
  \topic{zblue}{Cloud Logging}{%
    Each claim, after its processing via the Farm, produces logs, which
    are sent to the \href{https://papertrailapp.com}{PaperTrail} for
    monitoring purposes. On top of them, every exception is logged to
    \href{https://www.sentry.io}{Sentry}.}
  \topic{zblue}{XSL Sanitizer}{%
    Before a claim gets into the queue it is validated for its correctness
    by a series of XSL sanitizers. A claim may be rejected, for example,
    if it misspells the type or the login of the author.}
  \end{multicols}
  \end{multicols}
}


\slide{%
  Radar

  \begin{multicols}{2}
  \begin{tikzpicture}[scale=0.75, every node/.style={scale=0.75}]
    \component{slack-radar}{4,8}{Slack Radar}
    \component{github-radar}{8,8}{GitHub Radar}
    \component{telegram-radar}{12,8}{Telegram Radar}
    \component{claims}{8,4}{Claims}
    \draw [ln] (slack-radar) -- (claims);
    \draw [ln] (github-radar) -- (claims);
    \draw [ln] (telegram-radar) -- (claims);
    \node (slack) at (4,12) {\includegraphics[height=2em]{../images/slack.pdf}};
    \node (github) at (8,12) {\includegraphics[height=2em]{../images/github.pdf}};
    \node (telegram) at (12,12) {\includegraphics[height=2em]{../images/telegram.pdf}};
    \draw [ln] (slack) -- (slack-radar);
    \draw [ln] (github) -- (github-radar);
    \draw [ln] (telegram) -- (telegram-radar);
  \end{tikzpicture}
  \vfill
  \columnbreak
  \begin{multicols}{2}
    \topic{zblue}{Webhooks}{%
      Most radars receive events through pre-configured ``webhooks,''
      which send events in JSON format via RESTful API, which Zerocrat
      provides for them. For higher stability \href{http://www.rehttp.net}{ReHTTP}
      web service is used
      as a relay between chat platforms and radars.}
    \topic{zblue}{DSL}{%
      Each chat platform speaks its own language, while claims must have
      specifically unified types. The domain specific language claims speak
      is understood by all Stakeholders in the Farm.}
    \topic{zblue}{Decoupling}{%
      The way chat hubs are decoupled from Farm through the queue of claims
      gives a number of architectural benefits. First of all, the scalability
      is much higher, because the server doesn't need to reply immediately
      especially when the requests are coming in parallel. Second, longer
      response time is expectable by the user and the server can do many
      validating operations, no matter how complex is the request. The
      advantage of the chatbot architecture was explained in
      \href{https://www.yegor256.com/2015/11/03/chatbot-better-than-ui-for-microservice.html}{A Chatbot Is Better Than a UI for a Microservice}
      blog post.}
    \topic{zblue}{Modular Architecture}{%
      The number of external systems is growing and eventually Zerocrat will
      be integrated with a few dozens of them. Even though the complexity of each of
      them is not high, dealing with many of them at the same time is
      a serious challenge for the development team. Modular architecture
      simplified this task and makes the system more extendable.}
  \end{multicols}
  \end{multicols}
}

\slide{%
  Farm

  \begin{multicols}{2}
  \begin{tikzpicture}[scale=0.75, every node/.style={scale=0.75}]
    \component{stk}{8,8}{Stakeholders}
    \component{xocuments}{12,4}{Xocuments}
    \component{claims}{4,12}{Claims}
    \component{policy}{3,5}{Policy}
    \draw [ln] (claims) -- (stk);
    \draw [ln] (stk) -- (xocuments);
    \draw [ln] (stk) -- (policy);
  \end{tikzpicture}
  \vfill
  \columnbreak
  \begin{multicols}{2}
    \topic{zblue}{Groovy}{%
      Each stakeholder is a simple procedure, usually 100-200 lines long,
      in Groovy. Each stakeholder is responsible for its own narrow and isolated
      operation, which is processing of a claim and submitting new claims to
      the queue. Stakeholders ``talk'' to each other only through the queue
      of claims.}
    \topic{zblue}{Brigade}{%
      There is a collection of 150+ stakeholders (micro scripts), also known as ``brigade,''
      which perform individual management operations;
      the more complex is the Policy, the bigger the number
      of stakeholders.}
    \topic{zblue}{Configuration}{%
      The behavior of stakeholders is configured via the \href{https://www.zerocracy.com/policy.html}{Policy},
      an HTML document publicly hosted at \href{https://www.zerocracy.com}{zerocracy.com}.
      The document is a mix of plain English text and placeholders
      for configuration parameters. Once a change is submitted to the
      Policy, the behavior of stakeholders change immediately.}
    \topic{zblue}{XML}{%
      XML is the format of data for all project files (except the Ledger,
      which stays in the PostgreSQL database because of its very relational
      nature). The files are stored in AWS S3.}
    \topic{zblue}{XSD Schema}{%
      Each XML document in the storage is validated against its corresponding
      XSD Schema and is rejected if there are any issues. Thanks to this
      validation all documents in the storage are always valid and correct.}
    \topic{zblue}{XSLT}{%
      Each XML document in the storage has a corresponding XSLT stylesheet
      to convert it to HTML and render it in the dashboard.}
  \end{multicols}
  \end{multicols}
}

\tailslide

\end{document}
