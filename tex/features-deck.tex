\documentclass{deck}
\usepackage[american]{babel}
\usepackage{merriweather}
\begin{document}

\headslide{Key Features}

\slide{
  Zerocrat is a chatbot that automates routine project
  management operations for key software project stakeholders.

  \vspace{2em}\large
  The \colorbox{zgreen}{\color{white}{green}} features are implemented already, the
  \colorbox{zred}{\color{white}{red}} ones are still in the backlog.
}

\slide{
  Who Are the Stakeholders?

  \begin{multicols}{3}
  \topic{zblue}{Product Owner}{
    Funds the project,
    provides business requirements,
    accepts new releases from the Architect,
    approves hourly rates of Developers,
    reports bugs.
    There are usually one or two Product Owners
    in the project.
  }
  \topic{zblue}{Architect}{
    Boostraps the project,
    builds proof-of-concept,
    accepts bug reports and feature requests from Developers,
    releases new versions,
    regularly reports to the Product Owner.
    There is usually one Architect in the project, but more
    complex and risk-averse projects may want to assign a secondary
    Architect as a backup.
  }
  \topic{zblue}{Developer}{
    Implements features,
    fixes bugs,
    reports new bugs.
    There could be many Developers in a project, but a manageable and effective team
    usually include 10-30 people. Smaller teams tend to be slower than
    expected, while larger ones experience too many deadlocks
    and communication clashes.
  }
  \topic{zblue}{Tester}{
    Attempts to break the product,
    reports new bugs,
    accepts bug fixes from Developers.
    The best teams include as many Testers as they have Developers.
    The quality of the product only grows if the number of Testers is large.
  }
  \topic{zblue}{Quality Assurance}{
    Reviews the quality of communcation,
    makes sure Developers behave compliantly to the Policy,
    approves tickets closure.
    Usually, there is a single QA person in a project, while having a backup
    is advisable.
  }
  \end{multicols}

  \vspace{1em}\large
  A more detailed analysis of project roles can be found in this blog post:\\
  \href{https://www.yegor256.com/2016/07/10/software-project-roles.html}{\emph{Key Roles in a Software Project}}.
}

\slide{
  Operations
  \begin{multicols}{3}
  \topic{zgreen}{Assigns Tasks}{
    Zerocrat picks up a task from the entire scope of the
    project and decides which developer is the most suitable
    for it. There is a multi-factor election process explained in
    \href{http://www.zerocracy.com/policy.html\#3}{\S3}, which
    takes into account all visible and previously collected metrics
    of each developer, both in the current project and other
    projects in the entire platform. The election results become
    visible for the developer.
  }
  \topic{zgreen}{Rewards a Task}{
    When a \href{https://www.yegor256.com/2017/11/28/microtasking.html}{microtask}
    is completed Zerocrat pays the developer via
    \href{https://www.paypal.com}{PayPal},
    \href{https://www.upwork.com}{Upwork},
    \href{https://www.bitcoin.org}{Bitcoin},
    \href{https://www.zold.io}{Zold},
    or some other payment system suitable
    for worldwide micro transactions. Aside from the
    \href{https://www.yegor256.com/2014/09/24/why-monetary-awards-dont-work.html}{monetary reward}, the
    developer gets positive reputation points, which are a valuable driver
    of the gamification system.
  }
  \topic{zgreen}{Punishes for Delays}{
    When a task is not completed in time, which is usually ten days, Zerocrat
    takes the task away from the developer and attempts to assign it to someone
    else. A certain amount of negative reputation points go to the guilty
    delayer, as per \href{http://www.zerocracy.com/policy.html\#8}{\S8}.
  }
  \topic{zgreen}{Pays For Bugs}{
    According to \href{http://www.zerocracy.com/policy.html\#29}{\S29} Zerocrat
    rewards everybody who manages to submit a defect, which is approved
    by the Architect and got in to the project scope, be it a Tester, Developer,
    or even a Product Owner. The logic behind this was explained in
    \href{https://www.yegor256.com/2018/02/06/where-to-find-more-bugs.html}{\emph{More Bugs, Please}}
    blog post, with a few practical examples.
  }
  \topic{zgreen}{Pequests Code Reviews}{
    Each pull request, according to \href{http://www.zerocracy.com/policy.html\#27}{\S27},
    has to pass the first code review of a peer Developer and the second one
    of an Architect. Once a pull request is submitted Zerocrat finds the
    most suitable code reviewer, assigns to the task, and makes sure the
    rewards are paid accordingly.
  }
  \topic{zgreen}{Pays Bonuses}{
    There are situations when a Developer deserves a bonus, either in
    reputation points or monetary ones. For example, for completing
    a task faster than usual, as \href{http://www.zerocracy.com/policy.html\#36}{\S36}
    explains.
  }
  \topic{zgreen}{Registers Impediments}{
    Some tasks may have dependencies, which have to be resolved
    prior to the continuation of the task in hands. Per \href{http://www.zerocracy.com/policy.html\#9}{\S9}
    Zerocrat allows a Developer to put any task on hold for some time, to
    prevent it being taken away and re-assigned. Further, Zerocrat applies certain limitations
    to prevent the task of being held for too long and to prevent
    deadlocks.
  }
  \topic{zgreen}{Boosts a Budget}{
    As per \href{http://www.zerocracy.com/policy.html\#5}{\S5}, an Architect
    is allowed to increase the \href{https://www.yegor256.com/2018/01/09/micro-budgeting.html}{fixed budget}
    of a task a few times. Sometimes
    this manual ``boosting'' is required, especially when the task is too complex
    \emph{and} impossible to break down into smaller increments. Statistically speaking,
    boosting is required in less than 3\% of all tasks.
  }
  \topic{zgreen}{Pays Extra}{
    The Architect may instruct Zerocrat to send an arbitrary amount
    of cash to any Developer, when the Architect feels like this, as per
    \href{http://www.zerocracy.com/policy.html\#49}{\S49}. This
    feature is strongly discouraged to use, since it goes against the
    XDSD philosophy of money per result only. However, sometimes it's
    necessary to use this feature, especially in exceptional situations.
  }
  \topic{zgreen}{Rewards a Releases}{
    Every time a new version is released, Zerocrat rewards the Architect
    with a bonus, calculated by a formula explained in \href{http://www.zerocracy.com/policy.html\#54}{\S54}.
    The reasoning behind the formula
    is to motivate the Architect to release larger pieces of functionality
    more frequently.
  }
  \topic{zgreen}{Rewards a Review}{
    Every time a pull request is closed, the Architect gets a small bonus
    for the code review completed, per \href{http://www.zerocracy.com/policy.html\#28}{\S28}.
    The bonus motivates the Architect to stay on top all changes being merged.
  }
\end{multicols}}

\slide{
  People
  \begin{multicols}{3}
  \topic{zgreen}{Assigns a Mentor}{
    Zerocracy is an invite-only platform, which means that in order to
    become a Developer one has to be invited by someone already registered
    with us \emph{and} having a decent reputation score, as per
    \href{http://www.zerocracy.com/policy.html\#1}{\S1}. The inviter
    becomes a ``mentor'' who earns a certain
    \href{http://www.zerocracy.com/policy.html\#45}{``tuition fee''} as part of everything the
    student is making, until the student ``graduates'' after reaching
    a high enough reputation, as explained in
    \href{http://www.zerocracy.com/policy.html\#43}{\S43}.
  }
  \topic{zgreen}{Builds a Team}{
    Any Developer can freely join any project, provided the project Architect
    accepts the joining request. However, newbies are not even allowed to
    apply, until they spend some time in ``sandbox'' projects and build
    a decent reputation. Zerocrat manages the entire graduation process, as
    explained in
    \href{http://www.zerocracy.com/policy.html\#2}{\S2},
    \href{http://www.zerocracy.com/policy.html\#13}{\S13}
    \href{http://www.zerocracy.com/policy.html\#33}{\S33},
    and
    \href{http://www.zerocracy.com/policy.html\#35}{\S35}.
  }
  \topic{zgreen}{Promotes}{
    The Product Owner can promote a project, as in \href{http://www.zerocracy.com/policy.html\#26}{\S26},
    or an open vacation in it, as in \href{http://www.zerocracy.com/policy.html\#51}{\S51}.,
    using the Board of Zerocracy and all available communication channels
    with registered Developers. Thus, having a project in Zerocracy the
    Product Owner has free access to a large pool of engineers,
    which completely eliminates the necessity of recruting, hiring, interviewing
    and so on.
  }
  \topic{zred}{Invites}{
    Observing the situation in different projects Zerocrat decides
    which ones are in need of programmers and which Developers are most
    suitable for specific projects. Zerocrat regularly invites the right
    Developers into the right places.
  }
  \topic{zgreen}{Picks a Project}{
    Any Developer can freely join any project, provided the project Architect
    accepts the joining request. However, newbies are not even allowed to
    apply, until they spend some time in ``sandbox'' projects and build
    a decent reputation. Zerocrat manages the entire graduation process, as
    explained in
    \href{http://www.zerocracy.com/policy.html\#2}{\S2},
    \href{http://www.zerocracy.com/policy.html\#13}{\S13}
    \href{http://www.zerocracy.com/policy.html\#33}{\S33},
    and
    \href{http://www.zerocracy.com/policy.html\#35}{\S35}.
  }
  \topic{zred}{Defines Rates}{
    Looking at the results delivered by Developers Zerocrat makes
    a decisiion of who deserves a raise or a decline. The decision of
    Zerocrat is not disputable, but doesn't affect the effective rates
    in each project. It affects the marketable rate of each Developer,
    while the local rate stays in hands of the Product Owner.
  }
\end{multicols}}

\slide{
  Forecasts
  \begin{multicols}{3}
  \topic{zred}{Collects Metrics}{
    Zerocrat, via a constatly growing collection of adapters, collects
    metrics from the platforms where the code is being developed and programmers
    communicate, e.g. Git, \href{https://www.github.com}{GitHub},
    \href{https://www.atlassian.com/software/jira}{Jira},
    \href{https://jenkins.io/}{Jenkins},
    \href{https://www.sonarqube.org/}{SonarCube}, and so on. The
    projects get benchmarked by the metrics collected.
  }
  \topic{zred}{Re-Estimates}{
    Zerocract regularly requests randomly selected Developers to re-estimate
    the project, using the method of \href{http://www.technoparkcorp.com/innovations/scope-champions/}{Scope Champions}
    explained in the patent application
    \href{https://www.google.com/patents/US20100042968}{US 12/193,010}.
    Zerocrat uses provided estimates to predict the future of the project, both
    in terms of time and cost.
  }
  \topic{zred}{Determines Budget}{
    Using 1) project metrics, 3) benchmarking data across all projects in
    the platform, and 3) estimates provided by Developers, Zerocrat recalculates
    the future of the project and determines the budget, with certain
    accuracy and precision. The budget is recalculated multiple times a day,
    helping the Product Owner stay on top of the current project situation.
  }
  \topic{zred}{Develops Schedule}{
    Similar to the budget calculation, Zerocrat builds up project schedule,
    using statistical information and estimates. The schedule is also
    delivered to the Product Owner with certain accuracy and precision.
  }
  \topic{zred}{Benchmarks}{
    Zerocrat manages hundreds of projects at the same time and thus has
    direct access to the metrics in all of them. Using this data Zerocrat
    compares the current project with the statistical performance of others
    and makes recommendations to the Architect and the Product Owner to help
    them make preventive and responsive corrective actions.
  }
  \topic{zred}{Calibrates}{
    Zerocrat enables calibration and fine-tuning of key performance
    parameters, according to the learnings obtained during the course of
    the project and to the input of the Product Owner and the Architect.
  }
\end{multicols}}

\slide{
  Process
  \begin{multicols}{3}
  \topic{zgreen}{Enforces Policy}{
    There are multiple rules in the Policy, which have to be enforced in order
    to make sure the methodology works. For example, Zerocrat punishes
    for \href{http://www.zerocracy.com/policy.html\#6}{tasks refusal},
  }
  \topic{zgreen}{Reviews Quality}{
    The Quality Assurance reviews every completed micro-task and gives
    their ``quality verdict,'' which affects the rewarding formula
    Zerocrat uses to compensate the work of Developers and code reviewers,
    as in \href{http://www.zerocracy.com/policy.html\#30}{\S30}.
  }
\end{multicols}}

\slide{
  Scaffolding
  \begin{multicols}{3}
  \topic{zgreen}{Bootstraps a Project}{
    Each project in order to be started, as explained in \href{http://www.zerocracy.com/policy.html\#12}{\S12},
    has to be configured. The Architect earns a commission from all
    funds spent in the project.
  }
  \topic{zgreen}{Funds the Project}{
    The Product Owner may add funds to the project via credit card
    (we are using \href{https://www.stripe.com}{Stripe}). According to \href{http://www.zerocracy.com/policy.html\#21}{\S21}
    it is requires for a project to have funds on board before any tasks can be assigned
    to anyone. Funds are released only when tasks are completed. All
    unused funds, in case of project termination, can be refunded to the Product
    Owner, as per for \href{http://www.zerocracy.com/policy.html\#22}{\S22}.
  }
  \topic{zgreen}{Pauses a Project}{
    The Product Owner can put the entire project on hold at any moment, just
    by asking Zerocrat to do that, as explained in \href{http://www.zerocracy.com/policy.html\#24}{\S24}.
    The project can also be resumed any time. When the project is on pause,
    no funds will be spent, except for the tasks assigned before.
  }
  \topic{zgreen}{Vests Equity}{
    The Product Owner can rewards Developers not only with money,
    but equity of the project. Zerocrat automates this process and distributes
    equity in micro installments when tasks are getting completed,
    as explained in \href{http://www.zerocracy.com/policy.html\#37}{\S37}.
  }
  \topic{zgreen}{Destroy the Project}{
    The Product Owner can terminate the project and entirely remove
    the data from Zerocracy servers, as promised in
    \href{http://www.zerocracy.com/policy.html\#25}{\S25}.
  }
\end{multicols}}

\tailslide

\end{document}
