% SPDX-FileCopyrightText: Copyright (c) 2024-2025 Yegor Bugayenko
% SPDX-License-Identifier: MIT

\documentclass{main}
\pagecolor{black!3}
\usepackage[american]{babel}
\usepackage[ddmmyyyy]{datetime}
  \renewcommand{\dateseparator}{--}
\title{\includegraphics[scale=0.3]{../images/logo.pdf}\\
  Zold: A Fast Cryptocurrency for Micro Payments\\
  {\small\colorbox{white}{White Paper}}}
\author{Yegor Bugayenko\\
  \texttt{yegor256@gmail.com}\\
  \href{https://www.zold.io}{\texttt{www.zold.io}}\\[1em]
  \href{https://github.com/zold-io/papers/releases/tag/\zoldversion}{\texttt{\zoldversion}}}
\begin{document}
\date{July 6, 2019}

\maketitle
\begin{abstract}
In the last few years digital currencies have successfully demonstrated
their ability to become an alternative financial instrument in many
different markets. Most of the technologies available at the moment are
based on the principles of Blockchain architecture, including
dominating currencies like Bitcoin and Ethereum. Despite its
popularity, Blockchain is not the best possible solution for all scenarios.
One such example is for fast micro-payments.
Zold is an \emph{experimental} alternative that enables distributed transactions between
anonymous users, making micro-payments financially feasible and fast.
It borrows the ``proof of work'' principle from Bitcoin,
and suggests a different architecture for digital wallet maintenance.
Read \href{https://blog.zold.io}{our blog} for more details.
\end{abstract}

\section{Motivation}

Bitcoin, the first decentralized digital currency, was successfully released in January 2009~\parencite{nakamoto2008}.
In the following years ``a libertarian fairy tale'' and ``a simple Silicon Valley exercise in hype''
turned into ``a catalyst to reshape the financial system in ways that are more
powerful for individuals and businesses alike,'' according to \textcite{andreessen2014}.
Even though \textcite{cheah2015} argues that
``the fundamental value of Bitcoin is zero,''
it seems that ``the question is not whether Bitcoin has value; it already does.
The question is whether the efficiencies of a cybercurrency
like Bitcoin can be merged with the certainties of an honest central bank.''~\parencite{van2014}.

The core component of Bitcoin is Blockchain technology, which
``ensures the elimination of the double-spend problem, with the help
of public-key cryptography'' and ``coins are transferred by the
digital signature of a hash''~\parencite{pilkington2016}.
Very soon after Bitcoin was created, similar products were introduced,
which were also based on the principles of Blockchain, such as
Ethereum~\parencite{buterin2013}.

Even though Blockchain is a sound solution to the double-spending
problem, there could be other solutions,
including different ``proof-of-X'' alternatives.%
\footnote{%
  \url{https://goo.gl/aqzf2Q}:
  ``Proof-of-Burn'': instead of bringing the money together into computer equipment,
  the owner burns the coins by sending to an address where they are
  irretrievable. By doing this, the owner gets a privilege to
  mine on the system.
  ``Proof-of-Stake'': the coins exist from the start, and
  the validators get a reward in the form of transaction fees.
  ``Proof-Of-Capacity'': one pays with the hard drive space. The more
  dedicated hard drive space, the higher probability of mining
  the next block and earning a reward.
  ``Proof-of-Elapsed-Time'': one uses a Trusted Execution Environment or TEE
  to ensure a random looter production.
}
For example, \textcite{everaere2010} gave
a summary of them and introduced their own,
\textcite{boyen2016} described
``a truly distributed ledger system based on a lean graph of cross-verifying transactions'',
recently IOTA, a ``tangle-based cryptocurrency,'' was launched~\parencite{popov2017},
Hashgraph claims
to be ``the world's first mass-adopted public distributed ledger.''~\parencite{hashgraph},
and so on.

Zold is also a decentralized digital currency that maintains its ledgers
through an unpredictable amount of anonymous and untrustable server nodes, trying to guarantee
data consistency.
The development of Zold was motivated by the desire to overcome
two obvious disadvantages present in the majority of all existing
Blockchain-based cryptocyrrencies:

The first problem is that transaction processing is rather slow.%
\footnote{%
  \url{https://goo.gl/sWiAWc}:
  ``Current rates for Bitcoin processing
  speed is 7 transactions per second (tps) while Paypal handles
  an average of 115 tps and the VISA
  network has a peak capacity of 47,000 tps (though it currently needs 2000-4000 tps).''
}
\textcite{karame2012} says that ``Bitcoin requires tens of minutes to verify a transaction
and is therefore inappropriate for fast payments.''
It is inevitable, since
``processing speed is at odds with the security aspects of the underlying
proof-of-work based consensus mechanism'' according to~\textcite{kiayias2015}.
Ethereum, according to~\textcite{fekkes2018}, can process
``two times more transactions per second than Bitcoin is able to do,''
but this still is rather slow.

The second problem, as noted by~\textcite{popov2017}, is that ``it is not easy to get rid
of fees in the blockchain infrastructure since they serve
as an incentive for the creators of blocks.''
As per~\textcite{moser2015}, ``Bitcoin users are encouraged to
pay fees to miners, up to 10 cents (of USD) per transaction, irrespective of the
amount paid'' which especially hurts when transaction amounts are smaller than a dollar.
Moreover, according to~\textcite{kaskaloglu2014},
``an increase in transaction fees of Bitcoin is inevitable.''

Thus, the speed is low and the processing fees are high.
Zold was created as an attempt to resolve these two problems.

%%%%%%%%%%%%%%%%%%%%%%%%%%%%%%%%%%%%%%%%%%%%%%%%%%%%%%%%%%%%%%%%%%%%%%%%%%%%%%%%%%%%%%%%%%%%%%%%%%%
%%%%%%%%%%%%%%%%%%%%%%%%%%%%%%%%%%%%%%%%%%%%%%%%%%%%%%%%%%%%%%%%%%%%%%%%%%%%%%%%%%%%%%%%%%%%%%%%%%%
%%%%%%%%%%%%%%%%%%%%%%%%%%%%%%%%%%%%%%%%%%%%%%%%%%%%%%%%%%%%%%%%%%%%%%%%%%%%%%%%%%%%%%%%%%%%%%%%%%%
\section{Principles}

\textbf{No General Ledger}.
Unlike \emph{all} other crypto currencies, there is no central ledger in Zold.
Each wallet has its own personal ledger.
All transactions in each ledger are confirmed by
\href{https://en.wikipedia.org/wiki/RSA_(cryptosystem)}{RSA signatures} of their owners.
Section~\ref{sec:wallets} explains how this works.

\textbf{Proof of Work}.
Similar to many other digital currencies---including Bitcoin, Ethereum, Monero, Trinity, Plancoin, Dero,
and many others---Zold nodes find consensus by using the CPU power invested
by each of them to perform certain expensive and meaningless calculations that result in finding hash suffixes.
Section~\ref{sec:score} describes the algorithm being used.

\textbf{Detached Operations}.
To make a payment a user \emph{fetches} a wallet from the network,
makes as many payments from the wallet to other wallets as necessary,
and \emph{pushes} the wallet back to the network.
Thanks to that, there are no technical limitations to
the amount of transactions the network can process per second.
Section~\ref{sec:fetch} explains the details.

\textbf{Root Wallet}.
Zold is a pre-mined%
\footnote{%
  \url{https://goo.gl/QBhbcT}:
  ``A premine or instamine is where the developer or developers don't release
  the crypto currency in what can be considered a fair manner.
  Even Bitcoin can be considered to be instamined to a certain extent.
  A premine is where a developer allocates a certain amount of currency
  credit to a particular address before releasing the source
  code to the open community.''
}
digital asset, similar to Ripple,%
\footnote{%
  \url{https://goo.gl/XAtPH8}:
  ``When the Ripple network was created, 100 billion XRP was created.
  The founders gave 80 billion XRP to the Ripple Labs. Ripple Labs
  will develop the Ripple software, promote the Ripple payment system,
  give away XRP, and sell XRP.''
}
Cardano,
Stellar,%
\footnote{%
  \url{https://goo.gl/CnQQwA}:
  ``The stellar network started with 100 billion lumens.
  There is a 1\% p.a. inflation, hence the current total of roughly 103.5 billion lumens.
  About 18 billion lumens are on the market and the other
  85 is held by the stellar development foundation.''
}
EOS, NEO, Loki,%
\footnote{%
  \url{https://goo.gl/By5CR3}:
  ``Over 7 million Loki is held in escrow for the Founder,
  Advisor, and Seed allocations. The Founder and Advisor allocations
  follow a 12 month lockup schedule, where 25\% of each allocation is
  released every 90 days following mainnet launch. The allocations
  to Founders and Advisors are remuneration for services rendered to the
  LAG Foundation Ltd. The Seed allocation follows a similar schedule, with a
  30\% initial release and 20\% every 90 days until the final release of 10\%.''
}
and many others.
The only way to get ZLD is to receive it from someone else.
The root wallet belongs to the issuer and may have a negative balance.
All other wallets can only have positive balances.

\textbf{Taxes}.
Unlike many other payment systems, Zold doesn't require its users
to pay transaction fees. Instead, wallets have to pay regular ``taxes'' for their maintenance.
The amount of Taxes depends two factors: 1) the number
of transactions in a particular wallet and 2) the age of the wallet.
Section~\ref{sec:taxes} provides more details.

\textbf{No Trust}.
The network of communicating nodes maintain the wallets of Zold users.
Anyone can add a node to the network.
It is assumed that any node may contain corrupted data, either by mistake or intentionally.
Section~\ref{sec:remotes} explains how nodes communicate and rate each other.

\textbf{Open Source}.
Zold is a command line tool. Its entire code base is open source
and hosted on its GitHub \href{https://github.com/yegor256/zold}{yegor256/zold}
repository. There is also a primitive web wallet interface, with
GitHub OAuth login: \href{https://wts.zold.io}{wts.zold.io}.

\textbf{Capacity}.
One currency unit is called ZLD.
One ZLD by convention equals to $2^{32}$ \emph{zents} (4,294,967,296).
All amounts are stored as signed 64-bit integers.
Therefore, the technical capacity of the currency is 2,147,483,648 ZLD (two billion).%
\footnote{%
  To compare, the total supply of some crypto currencies is:
  Bitcoin: 21m BTC,
  Ripple: 100b XRP,
  Litecoin: 84m LTC,
  Cardano: 31b ADA,
  Stellar: 103b XLM,
  NEO: 100m NEO,
  Dash: 19m DASH.
}

\textbf{Hexspeak}.
A user's Wallet ID is a 16-digit hexadecimal number, which is not restricted by any formula.
A user may make up any number, even using
\href{https://en.wikipedia.org/wiki/Hexspeak}{Hexspeak}.

%%%%%%%%%%%%%%%%%%%%%%%%%%%%%%%%%%%%%%%%%%%%%%%%%%%%%%%%%%%%%%%%%%%%%%%%%%%%%%%%%%%%%%%%%%%%%%%%%%%
%%%%%%%%%%%%%%%%%%%%%%%%%%%%%%%%%%%%%%%%%%%%%%%%%%%%%%%%%%%%%%%%%%%%%%%%%%%%%%%%%%%%%%%%%%%%%%%%%%%
%%%%%%%%%%%%%%%%%%%%%%%%%%%%%%%%%%%%%%%%%%%%%%%%%%%%%%%%%%%%%%%%%%%%%%%%%%%%%%%%%%%%%%%%%%%%%%%%%%%
\section{Proof of Work}\label{sec:score}

The system consists of nodes (server machines), which maintain the data.
In order to guarantee data consistency among all distributed nodes
there has to be an algorithm of data segregation.
Corrupted data must be detected earlier and filtered out as quickly as possible.
Bitcoin employed this algorithm, which was originally introduced and labeled as
\emph{proof of work} by \textcite{back1997}.

Its fundamental principle is that each block of data must have a special
number attached to it, known as \emph{nonce}, which is rather difficult to calculate,
because it requires a lot of CPU power. It is assumed that at any moment
of time the majority of nodes in the network invest their CPU power into
calculating the nonces for uncorrupted data. If and when for any reason
certain data does get corrupted, the amount of CPU power that the corrupted part of the network
decides to invest into its nonces calculation would be smaller than what
the other part of the network invests into correct data. The latter part
will quickly dominate the former and the nodes with corrupted data will
be ostracized and eventually ignored~\parencite{nakamoto2008}.

Zold has borrowed elements of this principle, but has also modified it. Zold also requires
its nodes to invest their CPU power into meaningless and repetative
calculations in order to identify which part of the network they belong to:
corrupted or not. Each Zold node has to calculate its \emph{score},
which is indicative of the CPU power the node has invested into its calculation.

Similar to Bitcoin nonces, Zold nodes repeatedly calculate cryptographic hashes,
looking for consecutive zeros inside them. First, in order to calculate a score,
a node makes the \emph{prefix}, which consists of four parts,
separated by spaces:

\begin{enumerate}
\item The current timestamp in UTC, in \href{https://en.wikipedia.org/wiki/ISO_8601}{ISO 8601},
\item The host name or IP address, e.g. \dd{b2.zold.io},
\item The \href{https://en.wikipedia.org/wiki/Port_(computer_networking)}{TCP port} number,
\item The invoice.
\end{enumerate}

The host name and the IP are used in the prefix in order to prevent Sybil attack,
as explained by~\textcite{douceur2002},
where one server reuses the same score multiple times at different IP addresses
and/or TCP ports, representing itself as multiple servers.

For example, the prefix may look like this:

\begin{minted}{text}
2018-06-27T06:22:41Z b2.zold.io 4096 THdonv1E@abcdabcdabcdabcd
\end{minted}

Then, the node attempts to append any arbitrary text, which has to match
\dd{/[a-zA-Z0-9]+/} regular expression, to the end of the prefix and calculates
\href{https://en.wikipedia.org/wiki/SHA-2}{SHA-256 hash}
of the text in the hexadecimal format. For example, this would be the prefix
with the attached \dd{3a934b} suffix:

\begin{minted}{text}
2018-06-27T06:22:41Z b2.zold.io 4096 THdonv1E@abcdabcdabcdabcd 3a934b
\end{minted}

The hash of this text will be (pay attention to the trailing zeroes)%
\footnote{You can validate it at this online SHA-256 hash generator:
\url{https://goo.gl/QtHd9a}}:

\begin{minted}{text}
c9c72efbf6beeea13408c5e720ec42aec017c11c3db335e05595c03755000000
\end{minted}

The node attempts to try different sufficies until one of them produces
a hash that ends with a few trailing zeroes. The one above ends
with six zeroes.

When the first suffix is found, the score is 1. Then, to
increase the score by one, the next suffix has to be found, which
can be added to the previous hash
in order to obtain a new hash with trailing zeros. For example,
adding \dd{1421217}:

\begin{minted}{text}
c9c72efbf6beeea13408c5e720ec42aec017c11c3db335e05595c03755000000 1421217
\end{minted}

This new SHA-256 input produces the following output, which also
ends with six zeroes:

\begin{minted}{text}
e04ab4e69f86aa17be1316a52148e7bc3187c6d3df581d885a862d8850000000
\end{minted}

And so on. All hashes must have the same amount of trailing zeroes to make
the score valid.

The score is only valid when the starting time is earlier than
the current time, but not earlier than 24 hours ago. The \emph{strength} of the score
is the amount of trailing zeros in the hash. In the example above the
strength is six. The larger the strength, the more CPU power it takes to earn
the score. All nodes in the network must have the same strength of their scores.

There is no need for any node to aim for a strength that is bigger than the network requires.
However, a node must aim for larger values of the score, in order
to win a priority among all other nodes and earn more taxes,
as explained in Section~\ref{sec:taxes}.

Similar to Bitcoin network, which, according to~\textcite{hayes2017},
``automatically adjusts the difficulty
variable so that one block of bitcoins is found, on average, every ten minutes,''
Zold network may use the strength in order to calibrate itself.

%%%%%%%%%%%%%%%%%%%%%%%%%%%%%%%%%%%%%%%%%%%%%%%%%%%%%%%%%%%%%%%%%%%%%%%%%%%%%%%%%%%%%%%%%%%%%%%%%%%
%%%%%%%%%%%%%%%%%%%%%%%%%%%%%%%%%%%%%%%%%%%%%%%%%%%%%%%%%%%%%%%%%%%%%%%%%%%%%%%%%%%%%%%%%%%%%%%%%%%
%%%%%%%%%%%%%%%%%%%%%%%%%%%%%%%%%%%%%%%%%%%%%%%%%%%%%%%%%%%%%%%%%%%%%%%%%%%%%%%%%%%%%%%%%%%%%%%%%%%
\section{Wallets}\label{sec:wallets}

There is no central ledger in Zold, unlike many other digital currencies.
Instead, users have their own \emph{wallets} (any number of them) with their own ledgers inside.
Each wallet is an ASCII-text file with the name equal to the wallet ID.
For example, the wallet in the file \dd{12345678abcdef} may include
the following text:

\begin{minted}{text}
zold
1
12345678abcdef
AAAAB3NzaC1yc2EAAAADAQABAAABAQCuLuVr4Tl2sXoN5Zb7b6SKMPrVjLxb...

003a;2017-07-19T21:24:51Z; ffffffff9c0ccccd; Ui0wpLu7; 98bb82c81735c4ee; For services;SKMPrVj...
003b;2017-07-19T21:25:07Z; ffffffffffa72367; xksQuJa9; 98bb82c81735c4ee; For food;QCuLuVr4...
0f34;2017-07-19T21:29:11Z; 0000000000647388; kkIZo09s; 18bb82dd1735b6e9; -;
003c;2017-07-19T22:18:43Z; ffffffffff884733; pplIe28s; 38ab8fc8e735c4fc; For programming;2sXoN5...
\end{minted}

Lines are separated by either CR or CRLF.
There is a header and a ledger, separated by an empty line.
The header includes four lines:

\begin{enumerate}
  \item Network name, \dd{[a-z]\{4,16\}};
  \item Protocol version, \dd{[0-9]+};
  \item Wallet ID, a 64-bit unsigned integer in hexadecimal format;
  \item Public \href{https://en.wikipedia.org/wiki/RSA_(cryptosystem)}{RSA}
    key of the wallet owner, in \href{https://en.wikipedia.org/wiki/Base64}{Base64}.
\end{enumerate}

The ledger includes transactions, one per line. Each transaction line
contains fields separated by a semi-colon:

\begin{enumerate}
  \item \dd{id}: Transaction ID, an unsigned 16-bit integer, 4-symbols hex;
  \item \dd{time}: date and time, in \href{https://en.wikipedia.org/wiki/ISO_8601}{ISO 8601} format, 20 symbols;
  \item \dd{amount}: Zents, a signed 64-bit integer, 16-symbols hex;
  \item \dd{prefix}: Payment prefix, 8-32 symbols;
  \item \dd{bnf}: Wallet ID of the beneficiary, 16-symbols hex;
  \item \dd{details}: Arbitrary text, matching \dd{/[a-zA-Z0-9 -.]\{1,512\}/};
  \item \dd{signature}: \href{https://en.wikipedia.org/wiki/RSA_(cryptosystem)}{RSA} signature,
    684 symbols in \href{https://en.wikipedia.org/wiki/Base64}{Base64}.
\end{enumerate}

Transactions with positive amount don't have signatures.
Their IDs point to ID fields of corresponding beneficiaries' wallets.

The \dd{prefix} is a piece of text randomly selected from the RSA key
of the beneficiary wallet. It is used for security reasons, in order
to make impossible ``wallet masquerading'' (pushing a new wallet with the
same ID, but a different key).

The combination \dd{id}+\dd{bnf} must be unique throughout each wallet.

The \href{https://en.wikipedia.org/wiki/RSA_(cryptosystem)}{RSA}
signature is calculated using the private key of the
wallet and the following fields of transaction, separated by spaces:
\dd{bnf}, \dd{id}, \dd{time}, \dd{amount}, \dd{prefix}, \dd{bnf}, \dd{details}.

For example, this text may be used as a signing input:

\begin{minted}{text}
12345678abcdef 003a 2017-07-19T21:24:51Z ffffffff9c0ccccd Ui0wpLu7 98bb82c81735c4ee For services
\end{minted}

Each transaction takes 1284 symbols at most.

The order of transactions is not important, as long as the final balance is positive.

%%%%%%%%%%%%%%%%%%%%%%%%%%%%%%%%%%%%%%%%%%%%%%%%%%%%%%%%%%%%%%%%%%%%%%%%%%%%%%%%%%%%%%%%%%%%%%%%%%%
%%%%%%%%%%%%%%%%%%%%%%%%%%%%%%%%%%%%%%%%%%%%%%%%%%%%%%%%%%%%%%%%%%%%%%%%%%%%%%%%%%%%%%%%%%%%%%%%%%%
%%%%%%%%%%%%%%%%%%%%%%%%%%%%%%%%%%%%%%%%%%%%%%%%%%%%%%%%%%%%%%%%%%%%%%%%%%%%%%%%%%%%%%%%%%%%%%%%%%%
\section{Total Supply}\label{sec:supply}

Some cryptocurrencies, in order ``to protect against inflationary forces,''
algorithmically limit their total supply~\parencite{bohr2014}.
For example, Bitcoin halves the creation rate every 210,000 blocks,
which started from 50 BTC per block in 2009,
making it technically possible to ``mine'' the total amount
of 21 million until 2140~\parencite{iwamura2014}.

Ethereum, to the contrary, is infinite, although its total annual
supply is technically limited to 18 million ether. It is suggested
by Ethereum creators that potential token losses, caused by loss of keys and codes, death
of holders or misuse, ``will be compensated for by not having
a limit on the total amount of ether that will be created.''~\parencite{fekkes2018}.

In Zold, as explained in Section~\ref{sec:wallets}, any particular transaction can
transfer up to $2^{63}$ zents, which is, by convention, is $2^{32}$ ZLD.
Thus, technically, a summary of transactions in a wallet may be larger
than $2^{63}$. However, the software will complain and reject a wallet
if such an overflow happens. The minimum balance of a wallet is $-2^{63}$
and the maximum is $2^{63}$. Since there is only one wallet in the entire
network that can legally have a negative balance, the entire ``supply''
of Zold currency is roughly 2.15 billion ZLD.

%%%%%%%%%%%%%%%%%%%%%%%%%%%%%%%%%%%%%%%%%%%%%%%%%%%%%%%%%%%%%%%%%%%%%%%%%%%%%%%%%%%%%%%%%%%%%%%%%%%
%%%%%%%%%%%%%%%%%%%%%%%%%%%%%%%%%%%%%%%%%%%%%%%%%%%%%%%%%%%%%%%%%%%%%%%%%%%%%%%%%%%%%%%%%%%%%%%%%%%
%%%%%%%%%%%%%%%%%%%%%%%%%%%%%%%%%%%%%%%%%%%%%%%%%%%%%%%%%%%%%%%%%%%%%%%%%%%%%%%%%%%%%%%%%%%%%%%%%%%
\section{Taxes}\label{sec:taxes}

Each wallet must pay \emph{taxes} in order to be promoted by nodes.
The maximum amount of tax debt a node can tolerate is 1 ZLD. This means
that if the debt is smaller, all nodes must promote the wallet to their
remote nodes. If the debt is bigger, a node will reject the wallet,
which will make it impossible to make any new payments from it.

The amount of taxes to be paid is calculated by the following formula:

\begin{equation}
X = A \times F \times T.
\end{equation}

$A$ is the total age of the wallet,
which is calculated as the difference in hours between current time and
the oldest transactions in the wallet.
$T$ is the total number of transactions in the wallet.
$F$ is the fee per transaction/hour, which is equal to 1917 zents.

A one-year-old wallet with 4096 transactions will pay approximately 16 ZLD taxes annually
(1 ZLD equals to $2^{32}$ zents):

\begin{equation}
\begin{split}
X & = 365 \times 24 \times 4096 \times 1917z \\
  & = \text{35880960} \times 1917z \\
  & = \text{68783800320}z \\
  & \approx 16 \text{ZLD}.
\end{split}
\end{equation}

In order to pay taxes the owner of the wallet must select a remote
node from the network with a score of 16 or more. Then, it has to
take the invoice from the score, request the node to lock the score
for a minute, and send the payment of 1 ZLD or less
to that node. The score with exactly 16 suffixes
has to be placed into the details of the transaction,
prefixed by \dd{TAXES }.

The most active remote node will be selected as tax receiver.
It's up to the payer which node to select. This situation is similar
to Bitcoin, where ``a miner's chance of winning the
competition is (roughly, and with some caveats) equal to the proportion of the total computing
power that they control,'' according to~\textcite{foroglou2015}.

All tax payments inside a wallet must have unique scores.
Duplicate tax payments are ignored.

The following comparison demonstrates the cost of a thousand transactions
processing in different payment systems, including Zold. We have included both well-established systems and relatively new ones.
It is assumed that a Zold wallet is one year old
and 1 ZLD is worth \$1. Transaction fee costs in USD are approximate and are taken from the \href{https://bitinfocharts.com/}{BitInfoCharts} website. Where applicable, footnotes in the \textit{Price} column link to pages showing how transaction fees are calculated.

\begin{tabular}{lllr}
\hline
Cryptocurrency & Price & Date & Total cost \\
\hline
\href{https://getmonero.org/}{XMR} &
   Variable\footnotemark &
   June 28, 2018 &
   \$1070 \\
\href{https://bitcoin.org}{BTC} &
  43 sats/byte\footnotemark &
  June 26, 2018 &
  \$524 \\
\href{https://litecoin.org/}{LTC} &
  0.0001 LTC/kB\footnotemark &
  June 28, 2018 &
  \$94 \\
\href{https://www.dash.org/}{DASH} &
  0.00001--0.001 DASH\footnotemark &
  June 28, 2018 &
  \$58 \\
\href{https://www.ethereum.org/}{ETH} &
  Gas\footnotemark &
  June 28, 2018 &
  \$36 \\
\href{https://icon.foundation/}{ICX} &
  0.01 ICX\footnotemark &
  June 28, 2018 &
  \$15 \\
\href{https://ripple.com/xrp/}{XRP} &
   Variable\footnotemark &
   June 28, 2018 &
   \$12 \\
\href{https://www.bitcoincash.org/}{BCH} &
  1 sat/byte &
  June 21, 2018 &
  \$9 \\
\href{https://z.cash/}{ZEC} &
   0.0001 ZEC\footnotemark &
   June 28, 2018 &
   \$4 \\
\href{http://www.zold.io/}{ZLD} &
  32 zents/TH &
  \today &
  \$4 \\
\href{https://tron.network/}{TRX} &
  0.00001 TRX\footnotemark &
  June 28, 2018 &
  \$0.23 \\
\hline
\end{tabular}

% multiple \footnotemark causes problems with the counter. We have to increment the counter by -(n-1), where n is the number of \footnotemark we have.
\addtocounter{footnote}{-8}

\footnotetext{% XMR
  \emph{Monero Transaction Fee Calculator},
  Monero.How,
  \url{https://goo.gl/sEN6PE}.
}

% we must manually \stepcounter{footnote}
\stepcounter{footnote}
\footnotetext{% BTC
  Transaction Fee Info Charts,
  \url{https://goo.gl/ozsd63}.
}

\stepcounter{footnote}
\footnotetext{% LTC
  \emph{Understanding Default Minimum Fees},
  The Litecoin School of Crypto,
  \url{https://goo.gl/eUedqf}.
}

\stepcounter{footnote}
\footnotetext{% DASH
  \emph{Fees},
  Dash Official Documentation,
  \url{https://goo.gl/TKRrHy}.
}

\stepcounter{footnote}
\footnotetext{% ETH
  \emph{Ethereum Gas and Transaction Fees Explained!},
  BitFalls, May 2017,
  \url{https://goo.gl/B3ceVU}. For fairness, we are assuming only the cheaper fees for currency-only transactions, at 21000 Gas @ 4 Gwei.
}

\stepcounter{footnote}
\footnotetext{% ICX
  \emph{FAQ},
  ICON,
  \url{https://goo.gl/cwmKDD}.
}

\stepcounter{footnote}
\footnotetext{% ZEC
  \emph{Frequently Asked Questions},
  ZCash official website,
  \url{https://goo.gl/AHSwY8}.
}

\stepcounter{footnote}
\footnotetext{% XRP
  \emph{Calculating the Transaction Fee},
  Ripple Wiki,
  \url{https://goo.gl/CXQt3U}.
}

\stepcounter{footnote}
\footnotetext{% TRX
  \emph{Tron Main Net Launch},
  New Cryptocurrency---Tron,
  \url{https://goo.gl/TLWtmf}.
}

The comparison demonstrates that Zold is one of the most cost effective cryptocurrency payment system in the market.

%%%%%%%%%%%%%%%%%%%%%%%%%%%%%%%%%%%%%%%%%%%%%%%%%%%%%%%%%%%%%%%%%%%%%%%%%%%%%%%%%%%%%%%%%%%%%%%%%%%
%%%%%%%%%%%%%%%%%%%%%%%%%%%%%%%%%%%%%%%%%%%%%%%%%%%%%%%%%%%%%%%%%%%%%%%%%%%%%%%%%%%%%%%%%%%%%%%%%%%
%%%%%%%%%%%%%%%%%%%%%%%%%%%%%%%%%%%%%%%%%%%%%%%%%%%%%%%%%%%%%%%%%%%%%%%%%%%%%%%%%%%%%%%%%%%%%%%%%%%
\section{Remote Nodes}\label{sec:remotes}

Each node maintains a list of \emph{remote nodes} (their host names and TCP port numbers),
their scores and their availability information. When the node is first installed,
the list contains a limited amount of pre-defined addresses%
\footnote{\url{https://github.com/zold-io/zold/blob/master/resources/remotes}}.
The list is updated by manual user request and automatically in order to give priority
to high-score nodes and the nodes with the highest availability.
Moreover, the node adds new elements to the list by retrieving them from all other
available remote nodes.

The built-in mechanism focuses on the following factors of
remote node \emph{quality} (in order of importance):

\begin{enumerate}
  \item Visibility: the payer has to know the node;
  \item Availability: the amount of errors seen recently;
  \item Knowledgeability: the amount of nodes this node is aware of;
  \item Activity: the frequency of push requests the node originates;
  \item Score: the one reported during the most recent handshake.
\end{enumerate}

Every time a node receives an HTTP request from anyone, it reads the
`X-Zold-Score` HTTP header. The header, if it exists, may include
the text representation of a score of another node, which is making
the request. The request receiver validates the score and, if its
value is bigger than three, adds node coordinates (which are
available in the score) to the local list of remote nodes. Thanks to this
auto-discovery mechanism nodes become aware of each others presence
in the network.

Each node runs a ``reconnect'' procedure every minute, updating the list
of remote nodes and removing those, which have too low availability values.

%%%%%%%%%%%%%%%%%%%%%%%%%%%%%%%%%%%%%%%%%%%%%%%%%%%%%%%%%%%%%%%%%%%%%%%%%%%%%%%%%%%%%%%%%%%%%%%%%%%
%%%%%%%%%%%%%%%%%%%%%%%%%%%%%%%%%%%%%%%%%%%%%%%%%%%%%%%%%%%%%%%%%%%%%%%%%%%%%%%%%%%%%%%%%%%%%%%%%%%
%%%%%%%%%%%%%%%%%%%%%%%%%%%%%%%%%%%%%%%%%%%%%%%%%%%%%%%%%%%%%%%%%%%%%%%%%%%%%%%%%%%%%%%%%%%%%%%%%%%
\section{Fetch, Merge, and Propagate}\label{sec:fetch}

In order to see a wallet, it has to be \emph{fetched} from a number of remote
nodes. The nodes may provide different versions of the same wallet, either
because certain data is corrupted or because modifications were made to the same
wallet from different parts of the network. Each version retrieved from the
network is stored in a local \emph{copy} and gets a score assigned to it.
The score of the local copy is a summary of all scores of the nodes that
provided that copy. Let's say, there are 17 nodes in the network and they
provided three different copies of the wallet:

\begin{minted}{text}
copy-1: 78,090 bytes, 11 servers, 177 score
copy-2: 56,113 bytes, 4 servers, 69 score
copy-3: 97,132 bytes, 2 servers, 37 score
\end{minted}

The fetch operation ends at this point. The next step is to \emph{merge}
all three copies into the local one, if it exists. The algorithm of merging
is as follows:

First, the copy of the wallet into which we are merging is added to the list,
with the score of zero:

\begin{minted}{text}
copy-0: 55,991 bytes, 0 servers, 0 score
copy-1: 78,090 bytes, 11 servers, 177 score
copy-2: 56,113 bytes, 4 servers, 69 score
copy-3: 97,132 bytes, 2 servers, 37 score
\end{minted}

Then, the copy with the highest score is assumed to be the correct one,
which is the \dd{copy-1} in this example.

Then, all other copies, in the order of their scores, are merged into the
correct one, transaction by transaction, and the following rules are applied:

\begin{enumerate}
\item If the transaction already exists, it's ignored;
\item If the transaction is negative (spending money) and its ID is lower than
the maximum ID in the ledger, it gets ignored as a fraudulent one (``double spending'');
\item If the transaction makes the balance of the wallet negative, it is ignored;
\item If the transaction is negative and its signature is not valid, it is ignored;
\item If the transaction is positive and it's absent in the paying wallet
(which exists at the node), it's ignored; If the paying wallet doesn't exist at the node,
the transaction is ignored;
\item Otherwise, it gets added to the end of the ledger.
\end{enumerate}

When the merge process is complete, the modifications get \emph{propagated} to other wallets
available locally. Each transaction that has a negative amount is
copied to the ledger of their receiving wallets (with a reversed sign)
if it doesn't yet exist there.

%%%%%%%%%%%%%%%%%%%%%%%%%%%%%%%%%%%%%%%%%%%%%%%%%%%%%%%%%%%%%%%%%%%%%%%%%%%%%%%%%%%%%%%%%%%%%%%%%%%
%%%%%%%%%%%%%%%%%%%%%%%%%%%%%%%%%%%%%%%%%%%%%%%%%%%%%%%%%%%%%%%%%%%%%%%%%%%%%%%%%%%%%%%%%%%%%%%%%%%
%%%%%%%%%%%%%%%%%%%%%%%%%%%%%%%%%%%%%%%%%%%%%%%%%%%%%%%%%%%%%%%%%%%%%%%%%%%%%%%%%%%%%%%%%%%%%%%%%%%
\section{Invoice, Pay, and Push}\label{sec:pay}

To send money from one wallet to another, the owner of the sending wallet
has to add a negative transaction to it and sign it with the private RSA key.

Each transaction has to have a \emph{payment prefix} attached to it.
The prefix is a text block of 8-32 symbols randomly selected from the
text representation of the public key of the receiving wallet. This prefix
becomes a field in a transaction and participates in the body for the RSA
signature. Without this prefix it would be possible to steal a wallet by
replacing it with a new one, with a different pair of keys.

An \emph{invoice} is a combination of payment prefix and wallet ID separated
by the \dd{@} sign, for example:

\begin{minted}{text}
THdonv1E@abcdabcdabcdabcd
\end{minted}

Here, \dd{THdonv1E} is the payment prefix taken from the public key of
the wallet, and \dd{abcdabcdabcdabcd} is the wallet ID. Obviously, an
invoice is valid only if the prefix can be found in the public key of the wallet.

At any moment of time any node may decide to push a wallet to another node.
The receiving node accepts it, merges with the local version, and keeps it locally.
Then, it \emph{promotes} the wallet to all known remote nodes.

%%%%%%%%%%%%%%%%%%%%%%%%%%%%%%%%%%%%%%%%%%%%%%%%%%%%%%%%%%%%%%%%%%%%%%%%%%%%%%%%%%%%%%%%%%%%%%%%%%%
%%%%%%%%%%%%%%%%%%%%%%%%%%%%%%%%%%%%%%%%%%%%%%%%%%%%%%%%%%%%%%%%%%%%%%%%%%%%%%%%%%%%%%%%%%%%%%%%%%%
%%%%%%%%%%%%%%%%%%%%%%%%%%%%%%%%%%%%%%%%%%%%%%%%%%%%%%%%%%%%%%%%%%%%%%%%%%%%%%%%%%%%%%%%%%%%%%%%%%%
\section{RESTful API}\label{sec:api}

There is a limited set of RESTful API entry points in each node.
Each response has the following HTTP headers:
\begin{itemize}
\item \dd{Content-Type},
\item \dd{Content-Length},
\item \dd{X-Zold-Protocol}, and
\item \dd{X-Zold-Version}.
\end{itemize}

\dd{X-Zold-Protocol} is the protocol version.
It must be equal to the protocol version embedded in each wallet the
node works with, as explained in Section~\ref{sec:wallets}.

\dd{X-Zold-Version} tells the version of software a node is running.
Default semantic versions in triple format, e.g. \dd{0.13.46}, are used
by the default Ruby implementation in
\href{https://github.com/zold-io/zold}{zold-io/zold} GitHub repository.
For other implementations it is
advised to prefix the version with an implementation tag. For example,
a custom Java implementation made by Jeff may have a version \dd{jeff:0.4.19}.

\dd{GET /} is a home page of a node that returns
JSON/\href{https://www.w3.org/Protocols/rfc2616/rfc2616-sec10.html#sec10.2.1}{200}
response with the
information about the node. For example (other details may be added in
further versions):

\begin{minted}{json}
{
  "version": "0.6.1",
  "protocol: "1",
  "score": {
    "value": 3,
    "host": "b2.zold.io",
    "port": 4096,
    "invoice": "THdonv1E@0000000000000000",
    "suffixes": [ "4f9c38", "49c074", "24829a" ],
    "strength": 6,
    "time": "2018-06-20T05:22:54Z"
  }
}
\end{minted}

\dd{GET /remotes} returns the list of remote nodes known by the node,
in JSON/\href{https://www.w3.org/Protocols/rfc2616/rfc2616-sec10.html#sec10.2.1}{200}:

\begin{minted}{json}
{
  "version": "0.6.1",
  "protocol: "1",
  "all": [
    { "host": "b2.zold.io", "port": 4096 },
    { "host": "b1.zold.io", "port": 80 }
  ]
}\end{minted}

\dd{GET /wallet/<ID>} returns the content of the wallet, in
JSON/\href{https://www.w3.org/Protocols/rfc2616/rfc2616-sec10.html#sec10.2.1}{200}:

\begin{minted}{json}
{
  "version": "0.6.1",
  "protocol: "1",
  "body": "..."
}\end{minted}

The \dd{body} includes the entire content of the wallet file, according
to the format explained in Section~\ref{sec:wallets}.

If the wallet is not found, a
\href{https://www.w3.org/Protocols/rfc2616/rfc2616-sec10.html#sec10.4.5}{404}
HTTP response is returned.

If the client provided the pre-calculated MD5 hash of the wallet content in the
\href{https://www.w3.org/Protocols/rfc2616/rfc2616-sec14.html#sec14.26}{\dd{If-None-Match}}
HTTP header and it matches with the hash of the
content the node contains, a
\href{https://www.w3.org/Protocols/rfc2616/rfc2616-sec10.html#sec10.3.5}{304} HTTP response is returned.

\dd{PUT /wallet/<ID>} pushes the content of the wallet to the node. The
node responds either with
\href{https://www.w3.org/Protocols/rfc2616/rfc2616-sec10.html#sec10.2.3}{202} (if accepted),
\href{https://www.w3.org/Protocols/rfc2616/rfc2616-sec10.html#sec10.4.1}{400} (if the data is corrupted),
\href{https://www.w3.org/Protocols/rfc2616/rfc2616-sec10.html#sec10.4.3}{402} (if taxes are not paid),
or
\href{https://www.w3.org/Protocols/rfc2616/rfc2616-sec10.html#sec10.3.5}{304}
(if the content is the same as the one the node already has).

%%%%%%%%%%%%%%%%%%%%%%%%%%%%%%%%%%%%%%%%%%%%%%%%%%%%%%%%%%%%%%%%%%%%%%%%%%%%%%%%%%%%%%%%%%%%%%%%%%%
%%%%%%%%%%%%%%%%%%%%%%%%%%%%%%%%%%%%%%%%%%%%%%%%%%%%%%%%%%%%%%%%%%%%%%%%%%%%%%%%%%%%%%%%%%%%%%%%%%%
%%%%%%%%%%%%%%%%%%%%%%%%%%%%%%%%%%%%%%%%%%%%%%%%%%%%%%%%%%%%%%%%%%%%%%%%%%%%%%%%%%%%%%%%%%%%%%%%%%%
\section{Incentives}\label{sec:incentives}

Anonymous users will only participate in Zold and maintain
their nodes if they have enough financial motivation to do so. Simply
put, their expenses must be lower than the income they are getting in
the form of taxes. This Section analyzes the most obvious questions users
may have, regarding their motivation.

\subsection{To Stay Online}

What is the reason for a node to stay online and spend its CPU power
and network traffic? Each node is motivated by the
hopes that wallet owners will pay taxes to its invoices. The software
automatically decides to which node to pay taxes and the selection is
made by the availability criteria. The node which is most available
and visible will get the majority of tax payments.

Whether the tax payments in ZLD will become a strong enough incentive
to support the network is a separate problem, which exists in all
other decentralized cryptocurrencies. \textcite{iwamura2014} analyzes this
problem and concludes that ``if the Bitcoin price drops below
a threshold, the Bitcoin system as a whole may collapse.'' In other words,
the owners of nodes will be interested in keeping them online for as long
as the market price of ZLD taxes they collect is high enough to cover
their hardware and network traffic expenses.

\subsection{To Accept Wallets}

Why would a node accept push requests and spend its storage space
on the wallets coming in? If the node doesn't accept a push request,
its availability rating decreases and other nodes will stop paying
taxes to it.

\subsection{To Advertise Other Nodes}

What is the incentive to advertise other remote nodes via the \dd{/remotes} RESTful
entry point and why can't a node always return an empty list, expecting its clients
to always pay taxes to it? The software automatically prioritizes remote
nodes by the amount of other remote nodes it promotes. The longer the list a node
returns, the higher its chance to be at the top of the list.

\subsection{To Promote Wallets}

What is the incentive to promote wallets to remote nodes, spending network
traffic for this operation? Each node also ranks its remote nodes by the
amount of push requests they send. Thus, in order to stay on top of these rankings
each node is interested in pushing wallets further.


%%%%%%%%%%%%%%%%%%%%%%%%%%%%%%%%%%%%%%%%%%%%%%%%%%%%%%%%%%%%%%%%%%%%%%%%%%%%%%%%%%%%%%%%%%%%%%%%%%%
%%%%%%%%%%%%%%%%%%%%%%%%%%%%%%%%%%%%%%%%%%%%%%%%%%%%%%%%%%%%%%%%%%%%%%%%%%%%%%%%%%%%%%%%%%%%%%%%%%%
%%%%%%%%%%%%%%%%%%%%%%%%%%%%%%%%%%%%%%%%%%%%%%%%%%%%%%%%%%%%%%%%%%%%%%%%%%%%%%%%%%%%%%%%%%%%%%%%%%%
\section{Threats and Responses}\label{sec:threats}

It is obvious that a distributed system that consists of anonymous nodes
even theoretically can't be 100\% safe, reliable, secure and trustworthy.
Zold is not an exception. However, it's designed in an honest attempt
to mitigate all critical threats and make the system ``reliable enough.''
This Section summarizes the most import of those threads and explains
how Zold responds to them.

\subsection{Double Spending Attack}

It is possible to submit the same spending transaction to the same wallet
and then push it to two different nodes in different parts of the network.
They won't know about each other and will propagate those spending
transactions to other wallets. Both two owners of those money receiving
wallets will think that their money arrived, while only one of them is
a legit receiver, the other transaction is fraudulent.

This will happen, but very soon one part of the network will dominate the other
one, and one of the transactions will be rejected from the wallet, after
a number of merge operations in all nodes of the network. The receiver of the
money must be careful and always do the full fetch (from as many nodes
as possible) in order to guarantee safety of transactions.

\subsection{51\% Attack}

A group of nodes can combine their CPU power in order to win the consensus
algorithm and add fraudulent incoming transactions to a wallet.
The fetching node will trust the wallet ``as is'' and will think that the
balance of the wallet is larger than it actually is.

This may happen, but a fetching node may always re-validate the entire wallet,
by checking RSA signatures of all transactions. This will take some time, but will
provide an extra guarantee to the client.

\subsection{Fraudulent Tax Refunds}

Some nodes may resell their scores to their affiliated tax payers, and they
refund them some amount of taxes back. This will be profitable both for
the tax payers, since they will pay less taxes, and for the node owners,
since they will receive the payments anyway.

This scenario is indeed possible, but it is assumed that since tax payments are
supposed to be made in small increments and automatically, the majority of
clients won't be interested in this fraudulent scheme.

\subsection{Loss of Wallet}

A wallet is just a text file, which can easily be lost by its owner.

This indeed is possible, but is not a risk at all, since all wallets are
maintained by the network and anyone can easily pull any wallet back.
The only sensitive part is the private key of the wallet. If that file
is lost, the wallet can't pay anything anymore. This is the file all
wallet owners must keep safe.

\section{Conclusion}

The proposed architecture has two obvious advantages comparing to many other
similar solutions: it is low-latency and inexpensive. First, the speed of transaction processing
literally has no limits, because all payments are made locally, on users'
machines. Then, they push their wallets to the network and the network accepts
them. It is possible to make a large amount of payments in one batch and
submit them to the network in just one RESTful PUT request, as explained
in Section~\ref{sec:pay}. Second, the cost of transaction processing
is lower than most other payment systems can offer, as Section~\ref{sec:taxes}
demonstrated.

Despite the inevitable risks and threats any distributed payment system
has, Zold is a reliable alternative to Bitcoin, Ethereum, and
other existing platforms.

\section{Acknowledgements}

The first version of this document was created by Yegor Bugayenko in May 2018.
Since then there were many other contributors, who helped to make the text better,
corrected mistakes, suggested and made improvements. This is a non-complete list
of the most active participants of this collaborative process
(in alphabetic order of their first names):
%
Andrey Valyaev,
Michael Silver,
Norman Yeo,
Soh Yuan Chin.

\raggedright
\printbibliography%

\end{document}
